\cleardoublepage
\chapter{Conclusiones y recomendaciones}
\section{Conclusiones}
Se ensambló un instrumento con 116 ítems que permite evaluar de manera integral las características del síndrome depresivo. La variación ortogonal con doce factores muestra una estructura clínicamente plausible de dominios que evalúan a profundidad la sintomatología depresiva. Esta estructura es compatible con los modelos teóricos neuronales y con las descripciones fenomenológicas existentes. Creemos que la evaluación de la depresión mediante un instrumento de 12 dominios permite reunir de forma más minuciosa los hallazgos clínicos que las escalas clásicas con dos, tres o cuatro dominios. Consideramos que el enfoque metodológico utilizado derivó en un instrumento que podría ser usado en la investigación clínica de la depresión y evitar sesgos conceptuales que hasta el momento no podían evadirse. \\
Se evidenció una separación entre los dominios que evaluaban ideas depresivas e ideación suicida. La agrupación de esta última dentro del dominio Suicidio junto a la conducta suicida y los comportamientos autolesivos son objeto de investigaciones posteriores o análisis adicionales. Los dominios Sacrificio, Irritabilidad y Social agrupan pocos síntomas e insinúan la posibilidad de ampliación con otros ítems.\\
Se deben implementar técnicas estadísticas adicionales que permitan retiar ítems redundantes y refinar el instrumento actual.
\section{Recomendaciones}
Creemos que el instrumento obtenido evalúa de forma extensiva y minuciosa el síndrome depresivo e incluye síntomas que podrían ayudar a diferenciar perfiles o subtipos dentro de los trastornos depresivos. Sin embargo, es conveniente someter el instrumento actual a análisis posteriores que permitan refinarlo, eliminar ítems redundantes e identificar áreas susceptibles de ampliación.\\
La validación de este instrumento mediante teoría de respuesta al ítem puede ofrecer ventajas y aumentar la utilidad de la escala, permitiendo su uso en investigaciones futuras.\\