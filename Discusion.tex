\cleardoublepage
\chapter{Discusi\'{o}n}
Hemos presentado un proceso en el que a partir de la revisión de la literatura, la opinión de clínicos expertos y las percepciones de pacientes, se generó un instrumento que permite evaluar a profundidad la sintomatología depresiva. La metodología usada pretendió obtener un instrumento desde el análisis fenomenológico y alejado de las construciones teóricas existentes en el campo de los trastornos depresivos. Se buscó también distanciarse de los sistemas diagnósticos vigentes y permitir la exploración minuciosa y extensiva del síndrome depresivo. Este es, hasta donde tenemos conocimiento, el primer estudio que ha pretendido construir un instrumento de estas características usando métodos de análisis factorial.\\

Los instrumentos de medición usados en la clínica han sido construídos mediante la opinión de expertos y pocos han sido sometidos a análisis factoriales durante su construcción. Si bien las validaciones subsecuentes - usando teoría clásica de los test y en algunos casos teoría de respuesta al ítem - han mostrado resultados favorables, las estructuras factoriales encontradas son de entre 2 y 4 dominios, característicamente inespecíficos o que abarcan múltiples síntomas. Los instrumentos \emph{Beck, Hamilton, Zung y CES-D} suelen agrupar la mayor parte de la sintomatología depresiva bajo un gran factor Depresión y sólo alcanzan a diferenciar los síntomas somáticos, la ansiedad o en algunos casos los síntomas positivos \cite{Shafer2006}. Creemos que instrumentos más extensos y minuciosos son necesarios para avanzar en la investigación clínica de estos trastornos. Hallazgos en estudios previos que incluyeron ítems relacionados con depresión en análisis factoriales han identificado la necesidad de subdidivir la sintomatología depresiva y diferenciar entre las múltiples presentaciones clínicas de este síndrome \cite{Sanchez2011} \cite{VelasquezSuarez2013} \cite{RiveraViera2013}.\\

Existen, sin embargo, otros instrumentos que pretenden evaluar a profundidad los síntomas depresivos. El más cercano a esta investigación es el \emph{Depression Profile} realizado en Hungría en 2010 \cite{Faludi2010}. En la construcción de este inventario se tuvieron en cuenta los modelos teóricos de circuitos neuronales y se agruparon 111 síntomas en nueve dominios. Los ítems fueron obtenidos por la opinión de expertos y agrupados en una clasificación teórica de nueve dominios: 1. Ánimo Depresivo; 2. Pérdida del interés o el placer; 3. Cambios psicovegetativos; 4. Cambios psicomotores; 5. Cansancio y falta de energía; 6. Sensación de minusvalía y tendencias suicidas; 7. Decisión y ejecución; 8. Quejas somáticas; y 9. Ansiedad. Cada dominio tuvo entre 5 y 15 ítems cuya pertenencia al factor fue probada mediante métodos de análisis factorial. Las subescalas propuestas dentro del \emph{Depression Profile} guardan una relación cercana con las obtenidas en este estudio y muestran dominios similares.\\

Consideramos que el método usado para la construcción de este instrumento presenta ventajas significativas al disminuír el sesgo teórico y abarcar un gran número de manifestaciones sintomáticas agrupadas en 12 subescalas clínicamente plausibles.\\
\\
\section{Limitaciones}
Algunos dominios incluyen síntomas que aparentemente no estarían relacionados con los demás. Esto posiblemente obedezca a que en el listado inicial se incluyeron ítems que evaluaban síntomas somáticos o manifestaciones que a priori no estarían contenidos en el síndrome depresivo. Es posible que al incluirlos en el análisis factorial obtuvieran cargas altas sin que realmente estén evaluando síntomas depresivos.\\
La poca representación de pacientes con depresión y síntomas psicóticos posiblemente limita la distribución de los ítems que evaluaban delirios o alucinaciones en los dominios identificados, obteniendo cargas bajas y eliminándolos del instrumento definitivo.\\
Por último, la escala utilizada en el instrumento inicial que puntuaba la intensidad del ítem de cero a cinco no ha sido probada. Se desconoce si la calificación de intensidad resulta suficiente y si es la más apropiada para medir este tipo de sintomatología.\\