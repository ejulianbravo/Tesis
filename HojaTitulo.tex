%\newpage
%\setcounter{page}{1}
%%%%%%%%%%%%%%%%%%%%%%%%%%%%%%%%%%%%%%%%%%%%%%%%%%%%%%%%%%%%%%%%%%%%%%%%%%%%%%%%%%%%%%%%%%%%%%%%%%%%%%%%%%%%%%%%%%%%%%%%%%%%%%%%%%%%%%%%%%%%%%%%%%%%%%%%%%%%%% PRIMERA HOJA %%%%%%%%%%%%%%%%%%%%%%%%%%%%%%%%%
\begin{center}
\begin{figure}
\centering%
\epsfig{file=HojaTitulo/EscudoUN.eps,scale=1}%
\end{figure}
\thispagestyle{empty} \vspace*{2.0cm} \textbf{\huge
Desarrollo de un instrumento para medir s\'{i}ntomas depresivos}\\[6.0cm]
\Large\textbf{Edward Juli\'{a}n Bravo Naranjo}\\[6.0cm]
\small Universidad Nacional de Colombia\\
Facultad de Medicina, Departamento de Psiquiatr\'{i}a\\
Bogot\'{a}, D.C., Colombia\\
2015\\
\end{center}

%%%%%%%%%%%%%%%%%%%%%%%%%%%%%%%%%%%%%%%%%%%%%%%%%%%%%%%%%%%%%%%%%%%%%%%%%%%%%%%%%%%%%%%%%%%%%%%%%%%%%%%%%%%%%%%%%%%%%%%%%%%%%%%%%%%%%%

\newpage{\pagestyle{empty}\cleardoublepage}

\newpage
\begin{center}
\thispagestyle{empty} \vspace*{0cm} \textbf{\huge
Desarrollo de un instrumento para medir s\'{i}ntomas depresivos}\\[3.0cm]
\Large\textbf{Edward Juli\'{a}n Bravo Naranjo}\\[3.0cm]
\small Trabajo de grado presentado como requisito parcial para optar al t\'{i}tulo de:\\
\textbf{Especialista en Psiquiatr\'{i}a}\\[2.5cm]
Director:\\
M.D., MSc. Ricardo S\'{a}nchez Pedraza\\[2.0cm]
L\'{i}nea de Investigaci\'{o}n:\\
Investigaciones en Cl\'{i}nica Psiqui\'{a}trica\\
%Grupo de Investigaci\'{o}n:\\
%Nombrar el grupo en caso que sea posible\\[2.5cm]
Universidad Nacional de Colombia\\
Facultad de Medicina, Departamento de Psiquiatr\'{i}a\\
Bogot\'{a}, D.C., Colombia\\
2015\\
\end{center}

%%%%%%%%%%%%%%%%%%%%%%%%%%%%%%%%%%%%%%%%%%%%%%%%%%%%%%%%%%%%%%%%%%%%%%%%%%%%%%%%%%%%%%%%%%%%%%%%%%%%%%%%%%%%%%%%%%%%%%%%%%%%%%%%%%%%%%

\newpage{\pagestyle{empty}\cleardoublepage}

\newpage
\thispagestyle{empty} \textbf{}\normalsize
\\\\\\%

%\textbf{(Dedicatoria o un lema)}\\[4.0cm]

%%%%%%%%%%%%%%%%%%%%%%%%%%%%%%%%%%%%%%%%%%%%%%%%%%%%%%%%%%%%%%%%%%%%%%%%%%%%%%%%%%%%%%%%%%%%%%%%%%%%%%%%%%%%%%%%%%%%%%%%%%%%%%%%%%%%%%

\begin{flushright}
\begin{minipage}{8cm}
    \noindent
        \small
        \\[1.0cm]\\
    
        A mi madre y a los dem\'{a}s psicoestimulantes\\[1.0cm]\\
        \\[1.0cm]
        \\
\end{minipage}
\end{flushright}

\newpage{\pagestyle{empty}\cleardoublepage}

%%%%%%%%%%%%%%%%%%%%%%%%%%%%%%%%%%%%%%%%%%%%%%%%%%%%%%%%%%%%%%%%%%%%%%%%%%%%%%%%%%%%%%%%%%%%%%%%%%%%%%%%%%%%%%%%%%%%%%%%%%%%%%%%%%%%%%

\newpage
\thispagestyle{empty} \textbf{}\normalsize
\\\\\\%
\textbf{\LARGE Agradecimientos}
\phantomsection \addcontentsline{toc}{chapter}{\numberline{}Agradecimientos}\\\\
A mi profesor Ricardo S\'{a}nchez Pedraza por su gu\'{i}a en la realizaci\'{o}n del proyecto. A la estudiante de Internado Especial en Psiquiatr\'{i}a, Izara Mondrag\'{o}n por su ayuda en la recolecci\'{o}n de datos.\\

\newpage{\pagestyle{empty}\cleardoublepage}

%%%%%%%%%%%%%%%%%%%%%%%%%%%%%%%%%%%%%%%%%%%%%%%%%%%%%%%%%%%%%%%%%%%%%%%%%%%%%%%%%%%%%%%%%%%RESUMEN%%%%%%%%%%%%%%%%%%%%%%%%%%%%%%%%%%%%%

\newpage
\textbf{\LARGE Resumen}
\phantomsection \addcontentsline{toc}{chapter}{\numberline{}Resumen}\\\\


\textbf{Introducción:} El diagn\'{o}stico actual de depresi\'{o}n carece de estabilidad. Es posiblemente un s\'{i}ndrome inespec\'{i}fico que abarca m\'{u}ltiples presentaciones, con diferente curso y pron\'{o}stico. Hasta el momento no existe ning\'{u}n instrumento que eval\'{u}e ampliamente la sintomatolog\'{i}a depresiva y que permita eventualmente identificar subtipos dentro de los trastornos depresivos. Nos propusimos desarrollar un instrumento de medici\'{o}n que eval\'{u}e de forma integral la sintomatolog\'{i}a depresiva y que permita caracterizar y discriminar entre distintos subtipos de depresi\'{o}n.
\textbf{M\'{e}todos:} Se ensambl\'{o} un instrumento preliminar a partir de s\'{i}ntomas reportados en la literatura y se incluyeron s\'{i}ntomas adicionales a partir de entrevistas a profundidad con pacientes y de la opini\'{o}n de cl\'{i}nicos con experiencia. Se aplic\'{o} el instrumento a 248 pacientes con trastornos depresivos. Se usaron m\'{e}todos de estad\'{i}stica multivariada para estudiar la estructura de dominios, discriminar posibles agrupaciones sintom\'{a}ticas, eliminar s\'{i}ntomas redundantes y ubicar \'{a}reas susceptibles de ampliaci\'{o}n.
\textbf{Resultados:} El análisis factorial mediante doce (12) factores principales agrupó 116 ítems en doce dominios que se denominaron así: 1. Ideación Depresiva; 2. Anhedonia; 3. Bradipsiquia; 4. Autonómicos/Vegetativos; 5. Ansiedad; 6. Suicidio; 7. Somáticos; 8. Autosacrificio; 9. Tristeza; 10. Inercia/Hipobulia; 11. Irritabilidad; 12. Social. La mejor interpretabilidad se logró con una variación oblicua (Promax).
\textbf{Discusi\'{o}n:} Se ensambló un instrumento con 116 ítems que permite evaluar de manera integral las características del síndrome depresivo. La variación oblicua con doce factores muestra una estructura clínicamente plausible de dominios que evalúan a profundidad la sintomatología depresiva. Esta estructura es compatible con los modelos teóricos neuronales y con las descripciones fenomenológicas existentes. Es hasta el momento el primer instrumento de estas caracter\'{i}sticas desarrollado para evaluar a profundidad el síndrome depresivo.\\

%%%%%%%%%%%%%%%%%%%%%%%%%%%%%%%%%%%%%%%%%%%%%%%%%%%%%%%%%%%%%%%%%%%%%%%%%%%%%%%%%%%%%%%%%%%%%%%%%%%%%%%%%%%%%%%%%%%%%%%%%%%%%%%%%%%%%%

\textbf{\small Palabras clave:}\\
Psiquiatr\'{i}a, Depresi\'{o}n, Trastornos de Adaptaci\'{o}n, Trastorno Bipolar, Trastorno Depresivo Mayor, Trastorno Depresivo, Trastornos Mentales. (DeCS)\\\\


%%%%%%%%%%%%%%%%%%%%%%%%%%%%%%%%%%%%%%%%%%%%%%%%%%%%%%%%%%%%%%%%%%%%%%%%%%%%%%%%%%%%%%%%%%%%%%%%%%%%%%%%%%%%%%%%%%%%%%%%%%%%%%%%%%%%%%

\textbf{\LARGE Abstract}\\\\

\textbf{Background:} The current diagnosis of depression is probably a nonspecific syndrome that encompasses many forms, different courses and prognoses. There is no psycometric instrument to thoroughly evaluate the many symptoms associated with depression to identify different types of depressive disorders. We set out to develop a comprehensive instrument that included the integrity of depressive symptoms and that would characterize and differentiate between subtypes of depression. The aim of our study was to obtain an instrument to assess depressive symptoms.
\textbf{Methods:} We assembled a preliminary scale with symptoms found in classic phenomenologic descriptions and those included in depression rating scales. Symptoms obtained from interviews with patients and experienced clinicians were also included. The preliminary instrument was completed from the interviews with 248 patients diagnosed with depressive disorders. Multivariate statistical methods were used to characterize the factor structure of the instrument, find and eliminate redundant items and finding possible symptom clusters.
\textbf{Results:} Factor analysis by twelve (12) main factors grouped 116 items into twelve domains that were named as follows: 1. Depressive thoughts; 2. Anhedonia; 3. Bradypsychia; 4. Autonomic/Vegetative; 5. Anxiety; 6. Suicidal conduct; 7. Somatic pains and complaints; 8. Self-Sacrifice; 9. Depressed Mood; 10. Inertia/hypobulia; 11. Irritability; and 12. Social. The best interpretability was achieved with an oblique variation (Promax).
\textbf{Discussion:} An instrument with 116 items that comprehensively assess the characteristics of depressive syndrome was assembled. The oblique variation with twelve domains showed a factor structure that is clinically plausible and provides depth when assessing depressive symptoms. This structure is consistent with the neurocircuitry theory and with the existing phenomenological descriptions. It is so far the first instrument of its kind developed to evaluate in depth the depressive syndrome .\\


\textbf{\small Keywords:}.\\
Psychiatry; Depression; Adjustment Disorders; Bipolar Disorder; Depression, Bipolar; Depressive Disorder, Major; Depressive Disorder; Mental Disorders; Psychiatric Status Rating Scales. (MeSH)\\
