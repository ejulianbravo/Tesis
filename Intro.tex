\cleardoublepage
\chapter{Introducci\'{o}n}

La historia del diagn\'{o}stico psiqui\'{a}trico ha sido notable por su imperfecci\'{o}n, que se ve reflejada en los m\'{u}ltiples sistemas diagn\'{o}sticos, con clasificaciones, subdivisiones y especificadores que muchas veces se solapan entre s\'{i}. Dichas categor\'{i}as pretenden encasillar a los pacientes en unos cajones diagn\'{o}sticos, usualmente definidos arbitrariamente, que no reflejan la enfermedad y poco ayudan a escoger el tratamiento y conocer el pron\'{o}stico.\\
Los cuadros depresivos son sumamente heterog\'{e}neos, incluyendo un amplio rango de estados sindrom\'{a}ticos y subumbrales, con variaciones en su curso y en las caracter\'{i}sticas de cada episodio. Los continuos cambios en los criterios diagn\'{o}sticos y la observaci\'{o}n emp\'{i}rica de la existencia potencial de subtipos no permiten que exista un gold standard contra el cual validar nuevos instrumentos.\\
Tanto las descripciones cl\'{i}nicas encontradas en la literatura cient\'{i}fica, como los reportes subjetivos comprenden un espectro ampl\'{i}simo de s\'{i}ntomas depresivos, variaciones sutiles de estos y combinaciones que se extienden casi hasta el infinito. Se han descrito algunas categor\'{i}as que establecen m\'{a}s o menos unos l\'{i}mites entre estas variaciones, usualmente reflejando un criterio de severidad. Adem\'{a}s, algunos estudios han pretendido estudiar las caracter\'{i}sticas y las variaciones sintom\'{a}ticas dentro de los cuadros depresivos; usualmente partiendo del paradigma de la diferenciaci\'{o}n entre depresi\'{o}n bipolar y unipolar.\cite{Benazzi2005} \\
%
La evoluci\'{o}n de los sistemas diagn\'{o}sticos derivados de estas investigaciones contin\'{u}a sin cumplir las expectativas. La pobre diferenciaci\'{o}n entre episodios depresivos y la agrupaci\'{o}n de distintos fen\'{o}menos bajo el nombre de Trastorno depresivo mayor, la subvaloraci\'{o}n de s\'{i}ntomas, y el obviar caracter\'{i}sticas tan importantes como el curso, los patrones longitudinales, la historia familiar y la ciclicidad hacen que su utilidad pr\'{a}ctica sea limitada.\\
Posiblemente las debilidades de agrupar tantos fen\'{o}menos bajo un mismo nombre sean m\'{a}s evidentes en los estudios de estabilidad diagn\'{o}stica; definidos como el grado en el cual un diagn\'{o}stico se confirma en las evaluaciones subsecuentes \cite{Fennig1994}. Pacientes inicialmente diagnosticados como depresi\'{o}n unipolar que eventualmente cambian a alg\'{u}n trastorno bipolar, de ansiedad o psic\'{o}tico. \cite{Akiskal1995,Goldberg1995,Goldberg2001}.

Se hace evidente la necesidad de sistema de caracterizaci\'{o}n nosol\'{o}gica, neutral en cuanto a lo etiol\'{o}gico y que permita evaluar de forma integral los diferentes tipos de s\'{i}ntomas y su severidad, la existencia de eventos precipitantes, el impacto en el funcionamiento, las caracter\'{i}sticas de la personalidad y la presencia de comorbilidades. Sin un sistema que cumpla estos requisitos seguiremos sobrediagnosticando depresi\'{o}n, los subtipos depresivos seguir\'{a}n pasando desapercibidos, la inestabilidad diagn\'{o}stica se perpetuar\'{a} y el debate entre los factores biol\'{o}gicos, sociales y psicol\'{o}gicos continuar\'{a} sin resolverse.\\
\chapter{Marco Te\'{o}rico}
Los estados depresivos suelen caracterizarse por un enlentecimiento o disminución en casi todos los aspectos emocionales y del comportamiento. La velocidad del pensamiento, el discurso, la energía, sexualidad y la habilidad para sentir placer se ven mermados en  la mayoría de los cuadros.
La intensidad varía. Los síntomas se extienden desde un leve enlentecimiento físico y mental, con una mínima distorsión cognitiva y sensoperceptiva hasta profundos estados de estupor, delirios y alucinaciones, disminución del estado de alertamiento y extrema anhedonia.\\
Tanto las descripciones clínicas encontradas en la literatura científica, como los reportes subjetivos comprenden un espectro amplísimo de síntomas, variaciones sutiles de estos y combinaciones que se extienden casi hasta el infinito. Se han descrito algunas categorías que establecen más o menos unos límites entre estas variaciones, usualmente reflejando un criterio de severidad. 
La ‘melancholia gravis’ de Kraepelin, la melancolía delirante de Griesinger o locura depresiva contempla la presencia de síntomas psicóticos como un indicio de gravedad, siendo el delirium melancólico (‘delirious melancholia’) la que presenta la mayor afectación cognitiva y perceptiva, pudiendo llegar incluso al estupor depresivo, el máximo grado de retardo psicomotor encontrado en un cuadro depresivo.\\
%
\section{Manifestaciones Cl\'{i}nicas}
%
Si se definieran tres agrupaciones sintomáticas, cognición, sensopercepción y afecto, sería probablemente ésta última la que menos varíaría en el amplio espectro depresivo. Aún así, la irritabilidad y la ira suelen acompañar la tristeza característica.

Existen algunos estudios que han pretendido estudiar las características y las variaciones sintomáticas dentro de los cuadros depresivos; usualmente partiendo del paradigma de la diferenciación entre depresión bipolar y unipolar. Winokur y colaboradores (1969)   estudiaron la sintomatología y la evolución de 21 pacientes bipolares deprimidos (5 hombres, 16 mujeres, 33 episodios en total). Describieron el inicio de la depresión como abrupto en 5 episodios y como gradual en 28. El afecto se consideró melancólico o triste en prácticamente todos los pacientes (100 y 94 por ciento respectivamente), encontrándose desesperanza en aproximadamente la mitad de los pacientes (52\%). Tres cuartos de los pacientes deprimidos fueron descritos como irritables, lo que se corresponde con estudios más recientes en los cuales la irritabilidad es frecuente en la depresión bipolar.\cite{Goodwin2007,Deckersbach2004,Benazzi2005}.

De la misma forma, Winokur y colaboradores evaluaron síntomas cognitivos y sensoperceptivos, encontrado ideación de minusvalía y culpa en la mayoría de los pacientes (97 y 91\% respectivamente)1. Sólo el 9\% de los pacientes fallaron en reportar afectación cognitiva: 91\% reportó dificultad para concentrarse, disminución en la claridad o velocidad de pensamiento. La mitad reportó fallas mnésicas y se encontró una alta prevalencia de ideación suicida (82\%)  \cite{Goodwin2007}.

Otros estudios han encontrado una disminución en la producción verbal de los pacientes deprimidos, con algunas variaciones entre los grados de tangencialidad, circunstancialidad, perseveración, pérdida de asociaciones, ecolalia y bloqueos; afectaciones del pensamiento que en ocasiones se corresponden con los vistos en episodios maníacos.\cite{Ianzito1974,Andreasen1979,Andreasen1979a}

Los delirios y las alucinaciones ocurren en los cuadros depresivos más severos, tendiendo a reflejar características propias del estado de ánimo. Así, los delirios suelen incluír ideas de culpa y pecado, pobreza, quejas somáticas o autorreferencialidad. Las alucinaciones parecen ser menos frecuentes, predominando las auditivas con un contenido usualmente peyorativo o degradante para el paciente. La presencia o no de síntomas psicóticos en los cuadros depresivos parece permanecer estable en las recaídas o recurrencias, siendo el antecedente de síntomas psicóticos durante los episodios previos el mayor predictor de cuadros psicóticos futuros \cite{Charney1981,Frangos1983}. Charney y Nelson (1981) encontraron que el 95\% de los pacientes deprimidos con delirios habían tenido episodios depresivos con síntomas psicóticos previamente; hallazgo confirmado por los resultados de Helms y Smith \cite{Helms1983} y Lykouras \cite{Lykouras1985}. Aronson y colaboradores también encontraron una consistencia en las manifestaciones psicóticas a través de los episodios depresivos en pacientes bipolares y unipolares \cite{Aronson1988}.

Los síntomas somáticos, el grado de actividad y los cambios comportamentales han sido evaluados en otros estudios \cite{Casper1985}, encontrando la fatiga y el retardo psicomotor en aproximadamente tres cuartos de los pacientes deprimidos; las alteraciones del sueño, la pérdida del apetito y el deseo sexual son hallazgos así mismo frecuentes. Los cambios en el apetito y en el peso son más frecuentes en las mujeres \cite{Benazzi1999,Kawa2005}. La mayoría de los pacientes refiere quejas somáticas, y así mismo reportan una variación diurna del estado de ánimo (sintiéndose mejor en la noche y peor en la mañana).\\
%
\section{Diagn\'{o}stico}
%
La historia del diagnóstico psiquiátrico ha sido notable por su imperfección, que se ve reflejada en los múltiples sistemas diagnósticos, con clasificaciones, subdivisiones y especificadores que muchas veces se solapan entre sí. Dichas categorías pretenden encasillar a los pacientes en unos cajones diagnósticos, usualmente definidos arbitrariamente, que no reflejan la enfermedad y poco ayudan a escoger el tratamiento y conocer el pronóstico.\\
Los dos sistemas diagnósticos más usados, el de la Clasificación Internacional de Enfermedades (CIE) y el Manual Diagnóstico y Estadístico (DSM) presentan un conjunto de criterios usados para diagnosticar enfermedades o trastornos basándose en los síntomas obtenidos mediante una entrevista clínica \cite{AmericanPsychiatricAssociation2000, AmericanPsychiatricAssociation2013, Organization1993}.\\
Los conceptos fenomenológicos y los dogmas psiquiátricos se han visto reflejados en dichos sistemas diagnósticos; por ejemplo la diferenciación hecha entre las depresiones con presunta etiología psicosocial, llamadas ‘exógenas’ o ‘reactivas’ y las de supuesto origen biológico o ‘endógenas’; conceptos ya revaluados mediante la investigación empírica.\\
Se ha hecho necesario un sistema de clasificación nosológica, neutral en cuanto a lo etiológico y que permita evaluar los diferentes tipos de síntomas y su severidad, la existencia de eventos precipitantes, el impacto en el funcionamiento, las características de la personalidad y la presencia de comorbilidades. Sin un sistema que cumpla estos requisitos, el debate entre los factores biológicos, sociales y psicológicos continuará sin resolverse.
La evolución de estos esquemas diagnósticos continúa sin cumplir las expectativas. La pobre diferenciación entre episodios depresivos y la agrupación de distintos fenómenos bajo el nombre de ‘Trastorno depresivo mayor”, obviando características tan importantes como el curso, los patrones longitudinales, la historia familiar y la ciclicidad. 
Diversas críticas recaen sobre el sistema diagnóstico del DSM-IV, no sólo de los trastornos depresivos. El curso, los patrones longitudinales y la historia familiar en los trastornos afectivos recurrentes no están incluídos, a pesar de su evidente importancia. La falta de atención a dichos aspectos hace que el DSM resulte más confiable para el diagnóstico de un episodio aislado que para la enfermedad como conjunto. Por ejemplo, en estudios longitudinales de pacientes que sufren síntomas afectivos y psicóticos, los investigadores han notado que una parte importante sufre fluctuaciones sindromáticas; esto es, en un momento de su evolución pueden ser diagnosticados como esquizofrénicos y en otro como bipolares \cite{Marneros1988,Maj1989}.\\
Posiblemente las debilidades de agrupar tantos fenómenos bajo un mismo nombre sean más evidentes en los estudios de estabilidad diagnóstica; definidos como el grado en el cual un diagnóstico se confirma en las evaluaciones subsecuentes \cite{Fennig1994}. Pacientes inicialmente diagnosticados como ‘depresión unipolar’ que eventualmente cambian a algún trastorno bipolar \cite{Akiskal1995,Goldberg1995,Goldberg2001}.\\ 
Al realizar un seguimiento de 406 pacientes hospitalizados con diagnóstico de depresión unipolar, sólo el 60\% permaneció en dicha categoría diagnóstica al cabo de 20 años. Cada año, el 1.5\% de los pacientes inicialmente diagnosticados como unipolares fueron clasificados como bipolares (1\% TAB I, 0.5\% TAB II); 39,2\% cambiaron de diagnóstico en el transcurso del seguimiento \cite{Angst2005}.\\
En otro estudio que comparó la estabilidad del diagnóstico de acuerdo a los criterios usados, se encontró que el 91\% de los pacientes en los cuales se usó el CIE-10 mantuvieron dicho diagnóstico en los siguientes 3 años, mientras que únicamente el 78\% lo hicieron si se usaron los criterios del DSM-IV \cite{Amin1999}.\\
En otros estudios, entre el 40\% y el 60\% de pacientes con trastorno bipolar habían sido diagnosticados erróneamente con depresión unipolar tanto en contextos hospitalarios y de consulta externa \cite{Benazzi1999} \cite{Ghaemi1999} \cite{Ghaemi2000}. Estos hallazgos se relacionan con los encontrados en estudios de otros países distintos a EEUU. En un estudio español, se encontraron diagnósticos iniciales errados, siendo frecuentes los trastornos de la personalidad y la esquizofrenia \cite{Vieta1997}. En una evaluación cuidadosa de la historia de 108 pacientes evaluados en atención primaria, se demostró que el 44\% habían tenido síntomas de hipomanía, indicando que cerca de un 40\% podrían tener un diagnóstico errado de depresión unipolar \cite{Manning1997}.\\
Otra crítica que se hace al DSM es que, ya que cada ítem tiene el mismo peso sobre el diagnóstico final, no es posible que el uso de los criterios refleje el reconocimiento de patrones sintomáticos por el cual el clínico experto llega a un diagnóstico. Dichas limitaciones han sido expuestas por  van Praag:
\begin{quotation}
\emph{One can witness a standardized interview degenerating into a question-and-answer game: answers being taken on face value, not caring for the meaning behind the words, disregarding the as-yet-unspoken and oblivious to the emotional content of the communication… There is the danger of the desk researcher studying rating scale and standardized interview results rather than actual patients. These may be data collected not by himself, but by a research assistant with little psychiatric experience and training.} \cite{Praag1993}
\end{quotation}
Gran parte del problema y el error diagnóstico derivado de la estructura del DSM-IV recae sobre la separación entre unipolar y bipolar. Un trastorno depresivo mayor siempre será diagnosticado como unipolar, en tanto la clasificación como bipolar se basa únicamente en la presencia espontánea de un episodio maníaco o hipomaníaco. Al excluir la presencia de manía o hipomanía secundaria al uso de antidepresivos, y al presentar una estrecha definición de los episodios mixtos, el DSM-IV hace que sea mucho más probable diagnosticar depresión unipolar y menos frecuente el diagnóstico de trastorno bipolar \cite{Ghaemi2000}.\\
Desafortunadamente, en el caso de un paciente deprimido, la bipolaridad se basa exclusivamente en la referencia de un episodio de manía previo, haciendo que múltiples factores concernientes al paciente y al clínico impidan la obtención de dicha información y deriven en un diagnóstico errado \cite{Keitner1996}. La falta de introspección, la afectación de la memoria (característica del episodio depresivo) y las dificultades para recordar claramente los síntomas dificultan el diagnóstico \cite{Gershon1982}. Así mismo, la falta de reconocimiento de síntomas por parte de los clínicos \cite{Sprock1988}, el uso de prototipos diagnósticos \cite{Cantor1980}, la falta de experiencia y la simple falla en preguntar hacen que el mismo médico falle al momento de hacer un diagnóstico correcto.\\
\subsection{Instrumentos}
Los instrumentos de medición estandarizados proveen un lenguaje común para el uso en la clínica y la investigación. Al disminuir las diferencias entre cómo los clínicos registran los síntomas de los pacientes, las escalas contribuyen a compartir dicha información  y proveen una referencia para los diferentes observadores de un mismo fenómeno. 

Se han desarrollado múltiples instrumentos para evaluar las manifestaciones de los cuadros depresivos, que van desde la estimación de la severidad de un episodio particular, pasando por agrupaciones de síntomas aislados hasta la medición de factores de riesgo para desarrollar la enfermedad. Así pues, son de gran utilidad al momento de valorar la respuesta al tratamiento y realizar un seguimiento de la evolución del cuadro. Las escalas cuantitativas son de especial importancia en los estudios longitudinales, identificando la progresión y la resolución de patrones de síntomas, estudiando las características de los estados eutímicos y correlacionando los estados depresivos con otros aspectos cognitivos, del comportamiento, personalidad y neurobiológicos \cite{Goodwin2007}.

Se han delineado algunas categorías que pretenden clasificar los tipos de medición de síntomas, descritas por von Zerssen y Cording así: 1. Auto reporte, hecho por el paciente; 2. Reporte del observador, hechos por el clínico; 3. Análisis del comportamiento (incluyen análisis lingüísticos del discurso o de escritos; y 4. Mediciones objetivas, ya sea de actividades espontáneas (actividad física) o de reacciones usando una situacion estandarizada (test psicométrico objetivo) \cite{VonZerssen1978}.

La mayoría de estas escalas se han enfocado en la descripción de síntomas bajo agrupaciones (teóricas o estadísticas), bien desde la experiencia del paciente (auto reporte) o del observador, partiendo desde el conocimiento fenomenológico del cuadro afectivo; no existen instrumentos desarrollados específicamente para el estudio de los cuadros depresivos recurrentes.

\section{Aspectos Conceptuales y Metodol\'{o}gicos}

\subsection{Desaf\'{i}os conceptuales}

Existe una gran número de variables al momento de medir síntomas depresivos. Éstos tienden a ser altamente inestables, fluctuando rápidamente en el curso de días e incluso en cuestión de horas. En el contexto de los trastornos bipolares, se incluyen patrones temporales por definición, requiriendo la recolección longitudinal de datos. Los estados de ánimo pueden ser mixtos, incluyendo síntomas maníacos y depresivos, amén de otros elementos comórbidos como ansiedad, abuso de sustancias y trastornos de personalidad \cite{Goodwin2007}.

Los cuadros depresivos son sumamente heterogéneos, incluyendo un amplio rango de estados sindromáticos y subumbrales, con variaciones en su curso y en las características de cada episodio. Los continuos cambios en los criterios diagnósticos y la observación empírica de la existencia potencial de subtipos no permiten que exista un ‘gold standard’ contra el cual validar nuevos instrumentos.

Adicionalmente, la mayoría de las escalas pretenden incluir una multitud de síntomas clínicos, aún cuando los modelos etiológicos y la aparición de tratamientos específicos requieren que se preste una mayor atención al estudio de agrupaciones sintomáticas específicas (anhedonia, actividad motora, distraibilidad) que se encuentran pobrementes definidas y carecen de instrumentos de medición adecuados \cite{Baldessarini2003}. 
 

\subsection{Retos Metodol\'{o}gicos}

En adición a los aspectos conceptuales, existen numerosos requisitos psicométricos que deben cumplirse para garantizar una adecuada validez y fiabilidad del instrumento. La investigación ha mostrado la existencia de fallos en las herramientas usadas para la medición de síntomas, indicando que aún es necesario cerrar estas brechas \cite{Ronan2000}.

Establecer los objetivos en la medición representa un reto metodológico. El término \emph{depresión} es usado para denotar un estado del afecto, una miríada de síntomas, un síndrome clínico y una entidad diagnóstica ¿Qué pretendemos medir?. Los instrumentos diseñados para evaluar una constelación de síntomas depresivos, como el Inventario de Depresión de Beck \cite{Beck1961,Beck1988}, suelen ser usados erróneamente como herramientas diagnósticas, pretendiendo diagnosticar una enfermedad cuando el puntaje obtenido sobrepasa un punto de corte establecido. 
Múltiples escalas propenden por una puntuación de varias características sintomáticas, y aún así difieren significativamente en cuanto a los síntomas contenidos, el peso atribuído a cada uno y el énfasis en cada una de las dimensiones evaluadas (cognición, humor, síntomas somáticos) \cite{Snaith1993}.
Las metas, el propósito y las limitaciones de cada escala deberían ser claramente establecidos, permitiendo seleccionar un instrumento adecuado al objetivo propuesto y validando la toma de decisiones basadas en los resultados.