\cleardoublepage
\chapter{Materiales y M\'{e}todos}
%
\section{Dise\~{n}o}
Se llevó a cabo un estudio psicométrico, descriptivo y correlacional, de diseño transversal.\\
\subsection{Participantes}
Se evaluaron 252 pacientes con cualquier tipo de trastorno depresivo provenientes de instituciones públicas y privadas. Se entrevistaron pacientes de los hospitales Simón Bolívar E.S.E y la clínica Fray Bartolomé, así como de la consulta externa de psiquiatría de la Universidad Nacional de Colombia durante los años 2014 y 2015. Para incluir un paciente al estudio se requería que fuera mayor de edad, que hubiera sido evaluado previamente por un psiquiatra que le diagnosticara cualquier tipo de trastorno depresivo y su voluntad expresa en el consentimiento informado  Se incluyeron pacientes con Trastorno Depresivo Mayor, Episodios Depresivos, Trastorno Depresivo Mayor Recurrente, Trastorno Afectivo Bipolar Tipo I (que hubieran tenido episodios depresivos) y II, Distimia, Trastorno de la Adaptación con ánimo Depresivo, Trastorno Mixto de Ansiedad y Depresión, Duelo Complicado y Trastorno límite de la personalidad.\\Se contó con la aprobación de los comités científicos de las instituciones y con el aval del comité de ética de la Facultad de Medicina de la Universidad Nacional de Colombia.\\
\subsection{Instrumentos}
Se hizo una recolección de síntomas depresivos reportados en la literatura para generar un listado extenso que se depuró manualmente eliminando duplicados y síntomas evidentemente redundantes. Sin embargo, se conservaron variaciones sutiles de varios síntomas según figuraban en los instrumentos clínicos. Se incorporaron características clínicas adicionales a partir de dos entrevistas a profundidad con pacientes que hubieran tenido depresión y con la colaboración de clínicos con experiencia en la evaluación de estos trastornos.\\

Se revisaron 13 escalas de depresión de acuerdo a su relevancia clínica, popularidad, número de publicaciones y referencias del artículo original. Las escalas incluídas fueron las siguientes: \emph{OMS 5 item, Hamilton Rating Scale for Depression, Montgomery–Åsberg Depression Rating Scale (MADRS), Geriatric Depression Scale -  Yesavage, Zung Self-Rating Depression Scale, PHQ-9, Raskin Depression Scale, Beck Depression Inventory, Wechsler Depression Rating Scale, A clinically useful depression outcome scale (CUDOS), Goldberg Depression Scale, QUIDS y Calgary Depression Scale for Schizophrenia.}\\

Los síntomas recolectados se incluyeron en una matriz y se obtuvo un listado de 271 ítems a los que se incorporaron 8 síntomas de la colaboración de clínicos y 3 sugeridos por pacientes. También se incluyeron 7 ítems que evaluaban efecto Werther, antecedentes psiquiátricos personales y familiares. El listado de 289 ítems se depuró manualmente, eliminando síntomas duplicados y reduciendo la lista a 188 ítems. El instrumento depurado fue sometido a la evaluación por 2 profesores de Psiquiatría de la Universidad Nacional de Colombia expertos en el diagnóstico y tratamiento de trastornos depresivos.\\

Con el objetivo de facilitar el diligenciamiento del instrumento se separaron los síntomas en posibles dominios teóricos: Afecto, Cognitivo, Motor, Sensopercepción.
Se añadió una escala tipo Likert que puntuaba cada síntoma de cero a cinco \emph{(0:Nada; 1: Muy poco; 2: Poco; 3: Moderado; 4: Severo; 5: Muy Severo)}. Se verificó que no existieran ítems en los que fuera necesario invertir la escala.Finalmente, se añadieron campos para registrar la edad, el sexo, el uso de medicamentos, antecedentes personales de importancia, la condición de estar hospitalizado y una casilla para indicar si el voluntario se encontraba sintomático al momento de la entrevista. Se obtuvo un instrumento preliminar que evaluaba a profundidad la sintomatología depresiva, permitía calificar la severidad de cada síntoma y registrar datos demográficos y de antecedentes personales y familiares.\\
El instrumento completo que se aplicó a los participantes se encuentra en el \autoref{AnexoA} - Instrumento Preliminar.
\subsection{Procedimiento}
El instrumento final fue aplicado a todos los participantes mediante entrevista clínica semiestructurada realizada por clínicos con experiencia. Inicialmente se explicaba al voluntario las características del estudio y se obtenía el consentimiento para su participación. Las entrevistas tuvieron una duración entre 50 y 120 minutos y posteriormente el clínico diligenciaba el instrumento de acuerdo a la información obtenida.\\
\subsection{Análisis Estadístico}
Se utilizaron técnicas de estadística descriptiva para resumir las características de las variables demográficas y clínicas. El comportamiento de los puntajes de los ítems se describió mediante el uso de medianas y sus correspondientes rangos intercuartílicos como medidas de dispersión.\\
Para el componente analítico se usaron métodos de análisis multivariado. Se hizo un análisis factorial exploratorio utilizando métodos de factores principales y un análisis factorial confirmatorio de las estructuras encontradas mediante métodos de ecuaciones estructurales. Se aplicaron métodos de análisis multidimensional y de teoría de respuesta al ítem utilizando modelos de Rasch.\\
El análisis descriptivo y factorial se realizó con el programa R, versión 3.2.2. Para el manuscrito se usó el software \LaTeX\  y Texmaker v. 4.5.\\
\subsubsection{Análisis Factorial}
Los datos físicos obtenidos fueron codificados en una base de datos para realizar el análisis estadístico. Se realizó un análisis factorial exploratorio utilizando métodos de factores principales. Se evaluaron variaciones ortogonales (Varimax) y oblicuas (Promax) con seis y doce factores principales para identificar agrupaciones sintomáticas y dominios clínicamente plausibles. En cada variación se eliminaron los ítems con cargas factoriales menores a 0.3 en todos los factores y se agruparon los restantes teniendo en cuenta la carga factorial más alta y la mejor interpretación clínica.
Se compararon los resultados obtenidos mediante cada variación para escoger aquella con la mejor interpretabilidad.\\
\subsubsection{Análisis de la consistencia interna}
Se evaluaron los valores de alfa de Cronbach de la escala total, de los dominios encontrados en la etapa anterior, y del efecto del retiro de cada uno de los ítems del instrumento. Esto se efectuó con los mismos pacientes del grupo anterior.