\cleardoublepage
\chapter{Resultados} 
\section{Descripci\'{o}n de la muestra}
%%%%VAriables demográficas
Se aplicó el instrumento a 252 pacientes provenientes de IPS públicas y privadas. Se entrevistaron pacientes de los hospitales Simón Bolívar E.S.E y la clínica Fray Bartolomé, así como de la consulta externa de psiquiatría de la Universidad Nacional de Colombia.
Se excluyeron 4 pacientes, tres de ellos con diagnóstico de Esquizofrenia Paranoide (F200) y una paciente menor de edad (16 años), incluyéndose 248 pacientes en el análisis definitivo.
Se incluyeron pacientes de 18 a 87 años (\~{x}=30.5; IQR=24-49) de los cuales el 66.13\% eran mujeres (n=164). El 85.48\% (n=212) se encontraba sintomático al momento de la entrevista y el 60.48\% (n=150) estaba hospitalizado. El diagnóstico más frecuente fue Depresión Unipolar (50.4\%; n=143) seguido de Depresión Bipolar (22.18\%; n=55),  Trastornos de la Adaptación (n=35) y Distimia (n=25). 16 pacientes tenían algún trastorno de personalidad comórbido y 6 tenían asociado el consumo de psicoactivos. 165 pacientes (66.53\%) estaban recibiendo algún tipo de psicofármaco de los cuales el 67.2\% (n=111) recibían más de un medicamento; los más frecuentes ISRS (53.9\%; n=89) y antipsicóticos (47.3\%; n=78) seguidos de anticonvulsivantes (30.9\%; n=51).\\
\subsection{Análisis descriptivo}
%
%%%%% Medianas
%
El ítem que obtuvo las calificaciones más altas fue \emph{No estar feliz, contento o de buen humor (\~{x}=4; IQR=4-5)} seguido de otros ítems que evaluaron 
tristeza {\small[\emph{Tristeza (\~{x}=4; IQR=3-5), impulsos de llorar (\~{x}=4; IQR=3-5)}]}, 
irritabilidad {\small[\emph{Se siente continuamente irritado (\~{x}=4; IQR=3-5)}]}, 
anhedonia {\small[\emph{Disminución del tiempo dedicado a actividades placenteras (\~{x}=4; IQR=3-5), Las cosas le producen menos placer que antes (\~{x}=4; IQR=3-5), Ha dejado de disfrutar las cosas que antes disfrutaba (\~{x}=4; IQR=3-5), Pérdida del interés en el entorno (\~{x}=4; IQR=3-5)}]},
 hipobulia {\small[\emph{Se siente débil (\~{x}=4; IQR=3-5), Se siente sin energía (\~{x}=4; IQR=3-5), Le cuesta empezar a hacer algo (\~{x}=4; IQR=3-5)}} 
 e ideas de tipo depresivo {\small[\emph{Infravaloración (\~{x}=4; IQR=3-5), Piensa continuamente en sus errores (\~{x}=4; IQR=3-5), Ha perdido confianza en sí mismo (\~{x}=4; IQR=3-5), Siente que los demás están mejor (\~{x}=4; IQR=3-5), Es pesimista acerca del futuro (\~{x}=4; IQR=3-5), Autorreproche (\~{x}=4; IQR=2-5)}]}.\\
Las medianas más bajas correspondían a los ítems que evaluaban alteraciones sensoperceptivas {\small[\emph{Voces acusatorias (\~{x}=0; IQR=0), Alucinaciones simples (\~{x}=0; IQR=0-3), Ver sombras (\~{x}=0; IQR=0-3), Alucinaciones visuales amenazadoras (\~{x}=0; IQR=0-0) }]}, delirios {\small[\emph{Ideas delirantes de enfermedad (\~{x}=0; IQR=0), Ideas delirantes de ruina (\~{x}=0; IQR=0)}]}, el embotamiento afectivo {\small[\emph{No puede llorar aún cuando quiere hacerlo (\~{x}=1; IQR=0-3.25), Las cosas que solían irritarlo ya no lo hacen (\~{x}=1; IQR=0-4), Se siente incapaz de sentir rabia, tristeza, pena o placer (\~{x}=1; IQR=0-3)}]} y los antecedentes psiquiátricos familiares.\\ 
%
\subsubsection{Ansiedad}
%
%% Ansiedad
%
Los ítems que evaluaron ansiedad se encontraron con bastante frecuencia en los participantes {\small[\emph{No se siente tranquilo o relajado (\~{x}=4; IQR=3-5), Siente dificultad para relajarse (\~{x}=4; IQR=2-5), Se siente tenso (\~{x}=3; IQR=2-4), Se siente angustiado (\~{x}=3; IQR=2-4), Se siente ansioso sin saber por qué (\~{x}=3; IQR=1-4), Se siente nervioso (\~{x}=3; IQR=1-4), Siente que algo malo va a pasar (\~{x}=3; IQR=1-4), Se siente inquieto en la noche (\~{x}=3; IQR=1-4)]}} 
% Físicos
así como los síntomas físicos y autonómicos {\small[\emph{Se siente fatigado (\~{x}=4; IQR=2-5), Siente opresión en el pecho (\~{x}=3; IQR=2-5), Siente vacío en el estómago (\~{x}=3; IQR=1-4), Siente palpitaciones o falta de aliento (\~{x}=3; IQR=1-4), Xerostomía (\~{x}=3; IQR=1-4), Pesadez en las extremidades, espalda o cabeza (\~{x}=3; IQR=1-4), Dorsalgia, cefalalgia, mialgia (\~{x}=3; IQR=1-4)}]} 
% Sueño
y los relacionados con el sueño {\small[\emph{En el día se siente somnoliento (\~{x}=4; IQR=2-5), Le cuesta trabajo quedarse dormido (\~{x}=3; IQR=1-4), Se despierta antes de lo esperado (\~{x}=3; IQR=1-4), Se siente inquieto en la noche (\~{x}=3; IQR=1-4), Tiene que dormir en el día (\~{x}=3; IQR=1-4), No le da sueño (\~{x}=3; IQR=0-4), Se despierta fácilmente ante cualquier estímulo (\~{x}=3; IQR=1-5), Cuando se despierta siente como si no hubiera dormido (\~{x}=3; IQR=1-5), Se siente peor en la mañana (\~{x}=3; IQR=0-4)}]}.\\
%
%% Conducta motora y pensamiento
%
\subsubsection{Atenci\'{o}n y Pensamiento}
Los ítems que evaluaban la experiencia subjetiva de enlentecimiento cognitivo o dificultad para concentrarse tuvieron medianas superiores a aquellos que requerían la calificación por parte del clínico {\small[\emph{Se siente torpe (\~{x}=3; IQR=2-5), Dificultad para mantener la atención: leer, mantener una conversación (\~{x}=4; IQR=2-5), Reconoce estar deprimido y enfermo (\~{x}=4; IQR=1-5), No toma decisiones igual que antes (\~{x}=3; IQR=2-4), Tomar decisiones le es muy difícil o imposible (\~{x}=3; IQR=1-4), Siente que el pensamiento está más lento (\~{x}=3; IQR=1-4), Se siente físicamente enlentecido (\~{x}=3; IQR=1-5), Dificultad para poner en orden los pensamientos (\~{x}=2; IQR=1-4) v. Se siente lento en el diálogo (\~{x}=2; IQR=0-3,25), Le toma tiempo responder las preguntas (\~{x}=2; IQR=0-4), Diálogo difícil durante la exploración (\~{x}=1; IQR=0-3), Le cuesta o le es difícil responder a las preguntas (\~{x}=1; IQR=0-3)}]}.\\ 
%
Dentro de los ítems que evaluaban ideación depresiva, los más frecuentes estaban relacionados con ideas de minusvalía {\small[\emph{Infravaloración (\~{x}=4; IQR=3-5), Ha perdido confianza en sí mismo (\~{x}=4; IQR=3-5), Siente que los demás están mejor (\~{x}=4; IQR=3-5), Se siente incapaz (\~{x}=4; IQR=2-5), Se considera peor que los demás (\~{x}=4; IQR=2-5), Siente que tiene un peor aspecto que antes (\~{x}=4; IQR=2-5), Se siente inútil o sin valor (\~{x}=3; IQR=2-4)}]}, 
%
suicidio  {\small[\emph{Pensamientos sobre la posibilidad de morir (\~{x}=4; IQR=3-5)}}
%
y culpa {\small[\emph{Piensa continuamente en sus errores (\~{x}=4; IQR=3-5), Autorreproche (\~{x}=4; IQR=2-5), Siente que ha decepcionado a alguien más (\~{x}=4; IQR=2-4), Sentimientos de culpabilidad (\~{x}=3; IQR=2-4), Se culpa por todo lo malo que sucede (\~{x}=3; IQR=2-4)}]}. La presencia de ideas repetivas y rumiaciones depresivas también resultó frecuente (\~{x}=4; IQR=3-5).
\subsubsection{Apetito}
Si bien la Pérdida de apetito y la disminución de la ingesta fueron frecuentes (\~{x}=3; IQR=1-4), no lo fue así la pérdida de peso (\~{x}=1; IQR=0-3) ni la presencia de impulsos de comer entre comidas (\~{x}=1; IQR=0-2) o el aumento en la ingesta (\~{x}=1; IQR=0-2).\\
%
\subsubsection{Conducta suicida}
%
%% Suicidio
%
Se encontraron medianas altas en los ítems relacionados con la ideación y la conducta suicida {\small[\emph{Pensamientos sobre la posibilidad de morir (\~{x}=4; IQR=3-5), Ha pensado que sería mejor estar muerto (\~{x}=4; IQR=3-5), Deseo de estar muerto (\~{x}=4; IQR=2-5), Deseo de dormir y no despertar (\~{x}=4; IQR=2-5), Siente que no vale la pena vivir (\~{x}=3; IQR=2-4), Estar cansado de vivir (\~{x}=3; IQR=1-5), Comportamiento suicida o autolesivo (\~{x}=3; IQR=0-4), Ideación suicida (\~{x}=2; IQR=1-4), Amenazas de suicidio (\~{x}=1; IQR=0-4), Planes de suicidio (\~{x}=1; IQR=0-4)}]} y la desesperanza {\small[\emph{Es pesimista acerca del futuro (\~{x}=4; IQR=3-5), Siente que su vida está vacía (\~{x}=4; IQR=2-5), Cree que su situación no tiene solución (\~{x}=4; IQR=2-5)}]}.\\ 

%%% Listado completo
El listado completo de variables con sus medianas, rangos intercuartílicos y cuartiles se encuentra en el \autoref{AnexoB}.\\
%
\section{An\'{a}lisis factorial}
El gráfico de sedimentación de varianzas mostró dos posibles puntos de estabilización de la curva en 6 y 12 Factores principales (Ver curva de sedimentación de varianzas: \ref{Varianzas}).\\
\begin{figure}[h]
\caption{Curva de sedimentación de varianzas}
\label{Varianzas}
\centering
\includegraphics[width=0.8\textwidth]{varianzas}
\end{figure}
\\El análisis mediante seis factores principales demostró ser insuficiente ya que los factores 1 y 2 agrupaban un número excesivo de ítems. El primer factor además reunía síntomas relacionados con Anhedonia, Ideación Depresiva e Hipobulia, lo que no resultaba válido desde el punto de vista clínico. Las variaciones ortogonales y oblicuas con seis factores tampoco lograban una distribución plausible de los ítems. El listado completo de los ítems y sus cargas factoriales se encuentran en el Anexo \autoref{AnexoC}.\\
Mediante el análisis con doce (12) factores principales se obtuvo una agrupación de los ítems en 12 dominios clínicamente plausibles. El análisis inicial mostró entrecruzamiento de algunos ítems por lo que se analizaron y compararon variaciones ortogonales y oblicuas teniendo en cuenta cargas factoriales superiores a 0.4. Ambas variaciones mostraban agrupaciones clínicamente plausibles y agrupaban un número de ítems similar. La variación ortogonal (Varimax) obtenía 117 ítems aunque algunos dominios contenían ítems difíciles de interpretar. La mejor interpretabilidad se logró con una variación oblicua (Promax) que reunía 116 ítems. La comparación en número de ítems por dominio se resume en la Figura \ref{Red}. El listado completo de los ítems y sus cargas factoriales se encuentran en el Anexo \autoref{AnexoD}.
\begin{figure}[h]
\caption{Número de ítems por dominio según variación factorial}
\label{Red}
\centering
\includegraphics[width=1\textwidth]{red}
\end{figure}
\section{Identificaci\'{o}n de dominios}
El análisis factorial mediante doce (12) factores principales agrupó 116 ítems en doce dominios que se denominaron así: 1. Ideación Depresiva; 2. Anhedonia; 3. Bradipsiquia; 4. Autonómicos/Vegetativos; 5. Ansiedad; 6. Suicidio; 7. Somáticos; 8. Autosacrificio; 9. Tristeza; 10. Inercia/Hipobulia; 11. Irritabilidad; 12. Social.\\
El dominio \emph{Ideación Depresiva} agrupaba ítems que evaluaban a profundidad la presencia de cogniciones depresivas de minusvalía, culpa, autorreproche y pensamientos negativos. 
El dominio \emph{Anhedonia} agrupó la mayoría de ítems relacionados con la pérdida del placer, el interés o la disminución del tiempo destinado a las actividades placenteras. Los ítems relacionados con la sensación de enlentecimiento psíquico, fatigabilidad o la observación clínica de bradipsiquia, aumento en la latencia de las respuestas y fallas en atención se agruparon en el dominio \emph{Bradipsiquia}. El dominio \emph{Autonómicos/Vegetativos} incluyó la pérdida del apetito, de peso, disminución de la ingesta, xerostomía y palpitaciones. Los síntomas psíquicos de ansiedad como preocupaciones, intranquilidad y algunos físicos como opresión en el pecho, insomnio e inquietud motora se agruparon en el quinto dominio \emph{Ansiedad}.
Llamativamente los pensamientos sobre la posibilidad de morir o la ideación suicida explícita se agrupaban en el dominio \emph{Suicidio} junto a la conducta suicida y el comportamiento autolesivo. Bajo el dominio \emph{Somáticos} se agruparon ítems como cefalalgia, dorsalgia, mialgia, falta de energía y preocupaciones por problemas físicos. Los ítems relacionados con el comportamiento altruista y el sacrificio personal se agruparon en un dominio que se denominó \emph{Autosacrificio}. La tristeza, el llanto y el afecto depresivo persistente tuvieron cargas más altas en el noveno dominio que se denominó \emph{Tristeza}. El factor \emph{Inercia/Hipobulia} incluyó alteraciones volitivas, la dificultad para iniciar tareas y la somnolencia diurna. Sólo dos ítems que evaluaban irritabilidad obtuvieron cargas significativas y se agruparon junto a cuatro ítems inespecíficos; los ítems relacionados con la valoración negativa del entorno, los juicios negativos hacia otras personas y la disminución del contacto social se agruparon en el duodécimo factor \emph{Social}.\\
La lista de los ítems definitivos y sus cargas factoriales pueden verse en la Tabla \ref{promax}
\begin{center}
\small
\setlength\LTleft{0pt}
\setlength\LTright{0pt}
\begin{longtable}{@{\extracolsep{\fill}}lcc}
\caption[Promax]{Dominios Promax - Ítems y Cargas Factoriales} \label{promax} \\
\toprule
I. IDEACI\'{O}N DEPRESIVA & CARGA FACTORIAL & UNICIDAD \\
\midrule
Infravaloraci\'{o}n. & 0,5411 & 0,4484 \\
Piensa que merece ser castigado. & 0,7141 & 0,4325 \\
Delirios de culpa o pecado. & 0,5491 & 0,3566 \\
Siente que va a fracasar. & 0,4905 & 0,5482 \\
Se siente fracasado. & 0,4927 & 0,4046 \\
Es pesimista acerca del futuro. & 0,6163 & 0,4854 \\
Siente que no es maravilloso estar vivo. & 0,5104 & 0,5301 \\
Se siente in\'{u}til o sin valor. & 0,5338 & 0,577 \\
Cree que su situaci\'{o}n no tiene soluci\'{o}n. & 0,4889 & 0,4575 \\
Siente que los dem\'{a}s est\'{a}n mejor. & 0,4033 & 0,5501 \\
Siente verguenza de s\'{i} mismo. & 0,6042 & 0,4888 \\
Siente que se detesta a s\'{i} mismo. & 0,506 & 0,4526 \\
Se considera peor que los dem\'{a}s. & 0,6255 & 0,3504 \\
Siente que tiene un peor aspecto que antes. & 0,5593 & 0,4407 \\
Se siente viejo. & 0,5219 & 0,5416 \\
Siente que tiene un aspecto horrible. & 0,5053 & 0,4478 \\
Siente que su vida no podr\'{i}a ser peor. & 0,4865 & 0,5533 \\
Pensamientos sobre la posibilidad de morir. & 0,4622 & 0,493 \\
Estar cansado de vivir. & 0,5227 & 0,5525 \\
\midrule
II. ANHEDONIA & CARGA FACTORIAL & UNICIDAD \\
\midrule
Ha dejado de disfrutar las cosas que antes disfrutaba. & 0,7833 & 0,3425 \\
Ha perdido el inter\'{e}s por hacer las cosas. & 0,6334 & 0,4231 \\
P\'{e}rdida del inter\'{e}s en las actividades, aficiones o trabajo. & 0,6288 & 0,4715 \\
Las cosas le producen menos placer que antes. & 0,5912 & 0,5513 \\
P\'{e}rdida de inter\'{e}s en el entorno. & 0,5467 & 0,442 \\
Disminuci\'{o}n del tiempo dedicado a actividades placenteras. & 0,5164 & 0,5742 \\
La vida no est\'{a} llena de cosas que le interesan. & 0,51 & 0,4472 \\
Tristeza que no var\'{i}a de acuerdo a la situaci\'{o}n. & 0,5063 & 0,4654 \\
No obtiene una satisfacci\'{o}n aut\'{e}ntica de las cosas. & 0,5008 & 0,666 \\
Se siente sin ganas de hacer las cosas que antes hac\'{i}a. & 0,4728 & 0,4945 \\
Ha perdido confianza en s\'{i} mismo. & 0,4426 & 0,5356 \\
No toma decisiones igual que antes. & 0,4177 & 0,4657 \\
P\'{e}rdida de la l\'{i}bido. & 0,4108 & 0,5957 \\
Familiares con TAB. & -0,5786 & 0,4549 \\
\midrule
III. BRADIPSIQUIA & CARGA FACTORIAL & UNICIDAD \\
\midrule
Tomar decisiones le es muy dif\'{i}cil o imposible. & 0,4595 & 0,5128 \\
No puede hacer nada. & 0,484 & 0,574 \\
Siente que el pensamiento est\'{a} m\'{a}s lento. & 0,5145 & 0,403 \\
Est\'{a} tan cansado que no puede hacer nada. & 0,5496 & 0,5262 \\
Se cansa con cualquier cosa. & 0,5529 & 0,4002 \\
Se siente lento en el di\'{a}logo. & 0,5954 & 0,4078 \\
Se mueve, piensa y habla m\'{a}s lento. & 0,6021 & 0,3719 \\
Dificultad para poner en orden los pensamientos. & 0,6137 & 0,4549 \\
Incapacidad para mantener una conversaci\'{o}n sin esfuerzo. & 0,6697 & 0,426 \\
Di\'{a}logo dif\'{i}cil durante la exploraci\'{o}n. & 0,6878 & 0,4863 \\
Le cuesta o le es dif\'{i}cil responder las preguntas. & 0,7839 & 0,3462 \\
Le toma tiempo responder a las preguntas. & 0,8161 & 0,3115 \\
\midrule
IV. AUTON\'{O}MICOS-VEGETATIVOS & CARGA FACTORIAL & UNICIDAD \\
\midrule
Necesita obligarse a comer. & 0,7521 & 0,3847 \\
P\'{e}rdida del apetito. & 0,7441 & 0,3764 \\
Hiporexia. & 0,7438 & 0,3282 \\
P\'{e}rdida de peso. & 0,7433 & 0,4191 \\
La comida no sabe a nada. & 0,5514 & 0,4402 \\
Hiperventilaci\'{o}n. & 0,4528 & 0,4833 \\
Se preocupa por nimiedades. & 0,442 & 0,5555 \\
Xerostom\'{i}a. & 0,423 & 0,4384 \\
Siente palpitaciones o falta de aliento. & 0,4083 & 0,4324 \\
Come m\'{a}s que antes. & -0,4491 & 0,5351 \\
\midrule
V. ANSIEDAD & CARGA FACTORIAL & UNICIDAD \\
\midrule
Duerme m\'{a}s de lo habitual. & -0,5354 & 0,5236 \\
Tono de voz bajo. & -0,509 & 0,4367 \\
Tiene que dormir en el d\'{i}a. & -0,4074 & 0,6411 \\
No se siente tranquilo o relajado. & 0,405 & 0,6195 \\
Pasa la noche sin dormir. & 0,4357 & 0,505 \\
Siente opresi\'{o}n en el pecho. & 0,4543 & 0,4278 \\
Se siente ansioso sin saber por qu\'{e}. & 0,4665 & 0,6516 \\
Delirios de ruina. & 0,5258 & 0,543 \\
Le cuesta trabajo quedarse dormido. & 0,5983 & 0,4644 \\
Inquietud motora: juega con las manos, hala el cabello. & 0,63 & 0,4245 \\
No puede estar sentado. & 0,6711 & 0,4673 \\
Camina de un sitio a otro. & 0,762 & 0,3852 \\
\midrule
VI. SUICIDIO & CARGA FACTORIAL & UNICIDAD \\
\midrule
Intento de suicidio. & 0,7885 & 0,3235 \\
Amenazas de suicidio. & 0,7774 & 0,3665 \\
Ideaci\'{o}n suicida. & 0,7654 & 0,3972 \\
Ha pensado en lastimarse a s\'{i} mismo. & 0,6975 & 0,3996 \\
Comportamiento suicida o autolesivo. & 0,6935 & 0,3713 \\
Planes de suicidio. & 0,6908 & 0,4913 \\
Deseo de estar muerto. & 0,6261 & 0,3074 \\
Deseo de dormir y no despertar. & 0,5629 & 0,535 \\
Siente que no vale la pena vivir. & 0,4873 & 0,3753 \\
La sensaci\'{o}n es distinta al duelo. & -0,5937 & 0,412 \\
\midrule
VII. SOM\'{A}TICOS & CARGA FACTORIAL & UNICIDAD \\
\midrule
Hablar con voz triste. & 0,422 & 0,521 \\
Est\'{a} preocupado por problemas f\'{i}sicos o de salud. & 0,45 & 0,4252 \\
Evita tomar decisiones. & 0,5164 & 0,474 \\
Se siente sin energ\'{i}a. & 0,5501 & 0,5333 \\
Pesadez en las extremidades, espalda o cabeza. & 0,612 & 0,4676 \\
Dorsalgia, cefalalgia, mialgia. & 0,7517 & 0,3471 \\
\midrule
VIII. AUTOSACRIFICIO & CARGA FACTORIAL & UNICIDAD \\
\midrule
Se sacrifica por los dem\'{a}s. & 0,7976 & 0,3657 \\
Hace cosas por los dem\'{a}s a\'{u}n cuando le afecten. & 0,7783 & 0,371 \\
Siente que debe hacer cosas por los dem\'{a}s. & 0,7547 & 0,4097 \\
Familiares vistos por psiquiatra. & 0,4559 & 0,6161 \\
Familiares con depresi\'{o}n. & 0,4112 & 0,6938 \\
Expresi\'{o}n facial triste. & 0,4025 & 0,6627 \\
\midrule
IX. TRISTEZA & CARGA FACTORIAL & UNICIDAD \\
\midrule
Tristeza. & 0,526 & 0,4882 \\
Llanto f\'{a}cil. Llora por cosas que antes no lo hac\'{i}an llorar. & 0,56 & 0,5388 \\
Llora m\'{a}s que antes. & 0,6103 & 0,4332 \\
Llora continuamente. & 0,6053 & 0,4774 \\
Dolor de estar tan triste.  & 0,5599 & 0,4848 \\
No puede dejar de sentirse triste.  & 0,6469 & 0,4313 \\
Se siente afligido. & 0,4631 & 0,5445 \\
Esta es la peor tristeza que ha tenido. & 0,4494 & 0,3264 \\
Se siente desamparado. & 0,4335 & 0,4002 \\
\midrule
X. INERCIA-HIPOBULIA & CARGA FACTORIAL & UNICIDAD \\
\midrule
Se siente torpe. & 0,4104 & 0,5219 \\
Dificultad para empezar una tarea. & 0,6376 & 0,4071 \\
Dificultad para iniciar una tarea diaria simple. & 0,5374 & 0,5394 \\
Le cuesta empezar a hacer algo. & 0,7015 & 0,4151 \\
Tiene que obligarse para hacer algo. & 0,7084 & 0,4183 \\
En el d\'{i}a se siente somnoliento. & 0,4583 & 0,5474 \\
\midrule
XI. IRRITABILDAD & CARGA FACTORIAL & UNICIDAD \\
\midrule
Se culpar por todo lo malo que sucede. & 0,5167 & 0,4826 \\
El paciente reconoce estar deprimido y enfermo. & 0,5115 & 0,4988 \\
Ideas delirantes de enfermedad. & 0,4264 & 0,6509 \\
Niega estar enfermo. & -0,4348 & 0,5253 \\
Se molesta o se irrita m\'{a}s f\'{a}cil que antes. & -0,5605 & 0,514 \\
Se siente continuamente irritado. & -0,6138 & 0,5706 \\
\midrule
XII. SOCIAL & CARGA FACTORIAL & UNICIDAD \\
\midrule
Alucinaciones simples (sonidos de pasos, cadenas, golpes). & -0,4018 & 0,5683 \\
Se siente incapaz de sentir rabia, tristeza, pena o placer. & 0,4348 & 0,6075 \\
Tiende a culpar a los dem\'{a}s de lo que le pasa. & 0,5203 & 0,5231 \\
Ve m\'{a}s errores en los dem\'{a}s que antes. & 0,5362 & 0,4548 \\
P\'{e}rdida de inter\'{e}s en los dem\'{a}s.  & 0,6402 & 0,5293 \\
P\'{e}rdida de los sentimientos por familiares o personas cercanas. & 0,6746 & 0,5433 \\
\bottomrule
\end{longtable}
\end{center}